\def\layersep{1.33cm}
\def\nodesep{.75cm}
\def\nodesize{.35cm}

%\newcommand{\numdims}[0]{3}
%\newcommand{\numhidden}[0]{4}
%\newcommand{\upnodedist}[0]{0.6cm}
%\newcommand{\bardist}[0]{\hspace{-0.2cm}}

\begin{tabular}{c|c|c}
\hspace{-0.5cm}
\begin{tikzpicture}[shorten >=1pt,->,draw=black!50, node distance=\layersep]
    \tikzstyle{every pin edge}=[<-,shorten <=1pt]
    \tikzstyle{neuron}=[circle,fill=black!25,minimum size=17pt,inner sep=0pt]
    \tikzstyle{input neuron}=[neuron, fill=green!50];
    \tikzstyle{output neuron}=[neuron, fill=red!50];
    \tikzstyle{hidden neuron}=[neuron, fill=blue!50];
    \tikzstyle{annot} = [text width=4em, text centered]

    % Draw the input layer nodes
    \foreach \name / \y in {1,...,\numdims}
    % This is the same as writing \foreach \name / \y in {1/1,2/2,3/3,4/4}
        \node[input neuron, minimum size=\nodesize
        %, pin=left:Input \#\y
        ] (I-\name) at (0,-\nodesep*\y) {}; %{$x_\y$};

    % Draw the hidden layer nodes
    \foreach \name / \y in {1,...,\numhidden}
        \path[yshift=0.5cm]
            node[hidden neuron, minimum size=\nodesize] (H-\name) at (\layersep,-\nodesep*\y) {}; %{$\phi_\y$};

    % Draw the output layer node
    \foreach \name / \y in {1,...,\numdims}
    	\node[output neuron, minimum size=\nodesize
    	%,pin={[pin edge={->}]right:Output }
    	] (O-\name) at (2*\layersep,-\nodesep*\y) {};%{$f^{1}_\y$};

    % Connect every node in the input layer with every node in the
    % hidden layer.
    \foreach \source in {1,...,\numdims}
        \foreach \dest in {1,...,\numhidden}
            \path (I-\source) edge (H-\dest);

    % Connect every node in the hidden layer with the output layer
    \foreach \source in {1,...,\numhidden}
        \foreach \dest in {1,...,\numdims}
    	    \path (H-\source) edge (O-\dest);

    % Annotate the layers
    \node[annot,above of=I-1, node distance=\upnodedist] {Inputs};
    \node[annot,above of=H-1, node distance=\upnodedist] {Hidden};
    \node[annot,above of=O-1, node distance=\upnodedist] {Outputs};
\end{tikzpicture}
\bardist
&
\bardist
\begin{tikzpicture}[shorten >=1pt,->,draw=black!50, node distance=\layersep]
    \tikzstyle{every pin edge}=[<-,shorten <=1pt]
    \tikzstyle{neuron}=[circle,fill=black!25,minimum size=17pt,inner sep=0pt]
    \tikzstyle{input neuron}=[neuron, fill=green!50];
    \tikzstyle{output neuron}=[neuron, fill=red!50];
    \tikzstyle{hidden neuron}=[neuron, fill=blue!50];
    \tikzstyle{annot} = [text width=4em, text centered]

    % Draw the input layer nodes
    \foreach \name / \y in {1,...,\numdims}
    % This is the same as writing \foreach \name / \y in {1/1,2/2,3/3,4/4}
        \node[input neuron, minimum size=\nodesize
        %, pin=left:Input \#\y
        ] (I-\name) at (0,-\nodesep*\y) {};

    % Draw the hidden layer nodes
    \foreach \name / \y in {1,...,\numhidden}
        \path[yshift=0.5cm]
            node[hidden neuron, minimum size=\nodesize] (H-\name) at (\layersep,-\nodesep*\y) {};

    % Draw the hidden layer nodes
    \foreach \name / \y in {1,...,\numhidden}
        \path[yshift=0.5cm]
            node[hidden neuron, minimum size=\nodesize] (H2-\name) at (2*\layersep,-\nodesep*\y) {};


    % Draw the output layer node
    \foreach \name / \y in {1,...,\numdims}
    	\node[output neuron, minimum size=\nodesize
    	%,pin={[pin edge={->}]right:Output }
    	] (O-\name) at (3*\layersep,-\nodesep*\y) {};

    % Connect every node in the input layer with every node in the
    % hidden layer.
    \foreach \source in {1,...,\numdims}
        \foreach \dest in {1,...,\numhidden}
            \path (I-\source) edge (H-\dest);
            
    \foreach \source in {1,...,\numhidden}
        \foreach \dest in {1,...,\numhidden}
            \path (H-\source) edge (H2-\dest);            

    % Connect every node in the hidden layer with the output layer
    \foreach \source in {1,...,\numhidden}
        \foreach \dest in {1,...,\numdims}
    	    \path (H2-\source) edge (O-\dest);

    % Annotate the layers
    \node[annot,above of=I-1, node distance=\upnodedist] {Inputs};
    \node[annot,above of=H-1, node distance=\upnodedist] {Hidden};
    \node[annot,above of=H2-1, node distance=\upnodedist] {Hidden};
    \node[annot,above of=O-1, node distance=\upnodedist] {Outputs};
\end{tikzpicture}
\bardist
&
\bardist
\begin{tikzpicture}[shorten >=1pt,->,draw=black!50, node distance=\layersep]
    \tikzstyle{every pin edge}=[<-,shorten <=1pt]
    \tikzstyle{neuron}=[circle,fill=black!25,minimum size=17pt,inner sep=0pt]
    \tikzstyle{input neuron}=[neuron, fill=green!50];
    \tikzstyle{output neuron}=[neuron, fill=red!50];
    \tikzstyle{hidden neuron}=[neuron, fill=blue!50];
    \tikzstyle{annot} = [text width=4em, text centered]

    % Draw the input layer nodes
    \foreach \name / \y in {1,...,\numdims}
    % This is the same as writing \foreach \name / \y in {1/1,2/2,3/3,4/4}
        \node[input neuron, minimum size=\nodesize
        %, pin=left:Input \#\y
        ] (I-\name) at (0,-\nodesep*\y) {};

    % Draw the hidden layer nodes
    \foreach \name / \y in {1,...,\numhidden}
        \path[yshift=0.5cm]
            node[hidden neuron, minimum size=\nodesize] (H-\name) at (\layersep,-\nodesep*\y) {};

    % Draw the hidden layer nodes
    \foreach \name / \y in {1,...,\numhidden}
        \path[yshift=0.5cm]
            node[hidden neuron, minimum size=\nodesize] (H2-\name) at (3*\layersep,-\nodesep*\y) {};

    % Draw the output layer node
    \foreach \name / \y in {1,...,\numdims}
    	\node[output neuron, minimum size=\nodesize
    	%,pin={[pin edge={->}]right:Output }
    	] (O1-\name) at (2*\layersep,-\nodesep*\y) {};

    % Draw the output layer node
    \foreach \name / \y in {1,...,\numdims}
    	\node[output neuron, minimum size=\nodesize
    	%,pin={[pin edge={->}]right:Output }
    	] (O2-\name) at (4*\layersep,-\nodesep*\y) {};

    % Connect every node in the input layer with every node in the
    % hidden layer.
    \foreach \source in {1,...,\numdims}
        \foreach \dest in {1,...,\numhidden}
            \path (I-\source) edge (H-\dest);
            
    \foreach \source in {1,...,\numhidden}
        \foreach \dest in {1,...,\numdims}
            \path (H-\source) edge (O1-\dest);         
            
    \foreach \source in {1,...,\numdims}
        \foreach \dest in {1,...,\numhidden}
            \path (O1-\source) edge (H2-\dest);                

    % Connect every node in the hidden layer with the output layer
    \foreach \source in {1,...,\numhidden}
        \foreach \dest in {1,...,\numdims}
    	    \path (H2-\source) edge (O2-\dest);

    % Annotate the layers
    \node[annot,above of=I-1, node distance=\upnodedist] {Inputs};
    \node[annot,above of=H-1, node distance=\upnodedist] {Hidden};
    \node[annot,above of=O1-1, node distance=\upnodedist] {$\vf^{(1)}(\vx)$};
    \node[annot,above of=H2-1, node distance=\upnodedist] {Hidden};
    \node[annot,above of=O2-1, node distance=\upnodedist] {$\vf^{(1:2)}(\vx)$};

\end{tikzpicture} \\
\hspace{-0.5cm} a) One-hidden-layer MLP & 
b) Two-hidden layer MLP & 
\hspace{-0.3cm} c) Two-layer \gp{}: $\vy = \vf^{(2)} \left( \vf^{(1)}(\vx) \right)$
\end{tabular}
