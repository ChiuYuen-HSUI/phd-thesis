% ************************** Thesis Abstract *****************************
% Use `abstract' as an option in the document class to print only the titlepage and the abstract.
\begin{abstract}
Gaussian processes are a broad class of models,
% of functions
 useful for learning and forecasting in such domains as time series, geological formations, and physical dynamics.
The types of generalization these models can perform depends on the kernel,
%The main design decision when applying these models is choosing the kernel,
% or covariance function.
%The choice of kernel 
whose choice typically requires trial-and-error by experts,
limiting the power of Gaussian processes in practice.
In this thesis, we show how to automate this task.

We first detail the many types of structure that can be encoded through kernels, giving recipes to produce models with a wide variety of properties.
%, combining into one place many tricks of the trade, little-known extensions, and recipes 
%for combining kernels to produce a wide variety of properties.
For example, we show how kernels can be combined to produce priors over manifolds having the topologies of cylinders, torii, and M\"{o}bius strips, as well as their higher-dimensional analogues.

The main contribution of this thesis is to show how this open-ended space of models can be explored automatically.
We define a simple grammar over kernels, a search criterion (marginal likelihood), and a breadth-first search procedure.
Combining these, we present a procedure which takes a dataset, and outputs a detailed report with graphs and automatically-generated text illustrating the qualitatively different, and potentially novel, types of structure discovered in that dataset.
This system automates parts of the model-building and analysis currently performed by expert statisticians.

Kernel learning can also be seen as feature learning, a problem which has received intense interest in the last decade.
In particular, the practical success of deep neural networks has demonstrated the potential of automatic feature extraction.
Building on earlier work relating Gaussian processes and neural nets, we explore the natural extensions of these models to \emph{deep kernels} and \emph{deep Gaussian processes}.

%In addition to this automatic model-building procedure, w
We also explore several more extensions to Gaussian process models.
First, we examine the model class consisting of the sum of functions of all possible combinations of input variables.
We show a close connection between this model class and the recently-developed regularization method of \emph{dropout}.
Second, we combine Gaussian processes with the Dirichlet process to produce the \emph{warped mixture model} - an unsupervised clustering model with nonparametric cluster shapes, and a corresponding interpretable, low-dimensional latent space.
\end{abstract}
